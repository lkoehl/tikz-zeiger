\documentclass{standalone}
\usepackage{tikz}
\usetikzlibrary{calc,patterns,decorations.markings,decorations.pathmorphing,positioning}

\begin{document} 
\newcommand{\zeiger}[2]{
	\def \radius{#2 / 2}
	\def \angle{#1}
	\def \offset{0.5cm}
	\def \massSize{0.2cm}
		
	% Koordinatensystem
	\draw[->, gray] (-\radius-\offset,0) -- (\radius+\offset,0);
	\draw[->, gray] (0,-\radius-\offset) -- (0,\radius+\offset);
	
	% Winkelanzeiger
	\filldraw[fill opacity=0.3, color=blue!70]
	(0,0) -- (\offset,0mm) arc (0:\angle:\offset) -- (0,0) node[opacity=1] at (0.5*\angle:1.5*\offset) {\tiny $\angle$°};
		
	% Kreis
	\node[circle,draw,minimum size=\radius * 2, dashed, line width=0.7pt, gray] (c) at (0,0){}; 
		
	% Pfeil und Masse
	\draw[->, >=latex, line width=1pt](0:0) -- (\angle:\radius-\massSize);
	\draw[fill opacity=0.5,fill=red, red!60] (\angle:\radius) circle (\massSize);
}

\newcommand{\myfig}[3]{
	\tikzstyle{spring}=[thick,decorate,decoration={aspect=0.5, segment length=#1, amplitude=2mm,coil}]
	\tikzstyle{platform}=[fill,pattern=north east lines,draw=none,minimum width=2cm,minimum height=0.3cm]
	
	\coordinate (g) at (0,0);
	\coordinate (topspring) at (0,-0.2cm);
	\coordinate (bottomspring) at (0,{#2}); 
	\coordinate (pt2) at ($(bottomspring) + (0,-.2cm)$); 
	\coordinate (pt3) at ($(pt2) + (0,#3)$); 
	
	\node [platform,anchor=south] at (g)  {};
	\draw[very thick] (-1,0) -- (1,0);
	\draw [thick](topspring)--(g);
	\draw [spring] (bottomspring) -- (topspring);
	\draw [thick] (bottomspring) -- (pt2.north);
	\draw [opacity=0.5,fill=red, red!60] (pt3) circle (#3);
}

\begin{tikzpicture}
	\def \height{4cm}
	\def \xoffset{1cm}
	\zeiger{90}{\height}
	\begin{scope}[xshift=\height/2 + 2cm, yshift=\height/2 + 0.4cm + \xoffset]
		\myfig{0.7mm}{0 - \xoffset)}{-0.2cm}
	\end{scope}
\end{tikzpicture}
\begin{tikzpicture}
	\def \height{4cm}
	\def \xoffset{1cm}
	\zeiger{45}{\height}
	\begin{scope}[xshift=\height/2 + 2cm, yshift=\height/2 + 0.4cm + \xoffset]
		\myfig{1mm}{((-\height/2) * (1 - cos(45)) - \xoffset)}{-0.2cm}
	\end{scope}
\end{tikzpicture}
\begin{tikzpicture}
	\def \height{4cm}
	\def \xoffset{1cm}
	\zeiger{0}{\height}
	\begin{scope}[xshift=\height/2 + 2cm, yshift=\height/2 + 0.4cm + \xoffset]
		\myfig{2mm}{(-\height/2)  -\xoffset}{-0.2cm}
	\end{scope}
\end{tikzpicture}
\begin{tikzpicture}
	\def \height{4cm}
	\def \xoffset{1cm}
	\zeiger{270}{\height}
	\begin{scope}[xshift=\height/2 + 2cm, yshift=\height/2 + 0.4cm + \xoffset]
		\myfig{3.3mm}{(-\height) - \xoffset)}{-0.2cm}
	\end{scope}
\end{tikzpicture}

\end{document}